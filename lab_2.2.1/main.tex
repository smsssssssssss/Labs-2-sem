\documentclass[a4paper, 12pt]{article}
\usepackage[a4paper,top=1.5cm, bottom=1.5cm, left=1cm, right=1cm]{geometry}
\usepackage[utf8]{inputenc}
\usepackage{mathtext}
\usepackage{amsmath}
\usepackage{amsfonts}
\usepackage[english, russian]{babel}
\usepackage{indentfirst}
\usepackage{longtable}
\usepackage{graphicx}
\graphicspath{{pictures/}}
\DeclareGraphicsExtensions{.pdf,.png,.jpg}
\usepackage{natbib}
\usepackage{mathrsfs}
\usepackage[europeanresistors, americaninductors]{circuitikz}
\usepackage[justification=centering]{caption}
\usepackage{pgfplots}
\usepackage{tikz}
\usetikzlibrary{positioning}

\title{Лабораторная работа 2.2.1 Исследование взаимной диффузии газов}
\author{Ковешников Григорий, Соколова Маргарита, Б02-101}
\date{9 февраля 2022 г.}

\begin{document}
\maketitle

	\begin{center}
		{\Large Определение теплоты испарения жидкости}
	\end{center}
\noindent \textbf{Цель работы:} \\
\noindent  Регистрация  зависимости  концентрации   гелия в воздухе от времени с помощью датчиков теплопроводности при разных начальных давлениях смеси газов; определение коэффициента диффузии по результатам измерений.\\

\noindent \textbf{В работе используются:} \\
\noindent измерительная установка; форвакуумный насос; баллон с газом (гелий); манометр; источник питания; магазин сопротивлений; гальванометр; секундомер.

\section*{Теоретические сведения}
\noindent Рассмотрим процесс выравнивания концентрации. Пусть концентрации одного из компонентов смеси в сосудах $V_1$ и $V_2$ равны $n_1$ и
$n_2$. Плотность диффузионного потока любого компонента (т. е. количество вещества, проходящее в единицу времени через единичную поверхность) определяется законом Фика:
$$j=-D\frac{\partial n}{\partial x},$$ 

\noindent где $D$ — коэффициент взаимной диффузии газов, а $j$ - плотность потока частиц.\\

\noindent В нашем случае ввиду того что, а) объем соединительной трубки мал по сравнению с объемами сосудов, б) концентрацию газов внутри каждого сосуда можно считать постоянной по всему объему. Диффузионный поток в любом сечении трубки одинаков. Поэтому, 
$$J=-DS\frac{n_1-n_2}{l}.$$

\noindent Обозначим через $\Delta n_1$ и $\Delta n_2$ изменения концентрации в объемах $V_1$ и $V_2$ за время $\Delta t$. Тогда $V_1 \Delta n_1$ равно изменению количества компонента в объеме $V_1$, а $V_2 \Delta n_2$ — изменению количества этого компонента в $V_2$. Из закона сохранения вещества следует, что $V_1n_1+V_2n_2 = const$, откуда $V_1 \Delta n_1 = -V_2\Delta n_2.$ Эти изменения происходят вследствие диффузии, поэтому:
$$V_1\Delta n_1=-V_2\Delta n_2.$$

\noindent С другой стороны $V_1\Delta n_1=J\Delta t$ и $V_1\frac{dn_1}{dt}=-DS\frac{n_1-n_2}{l}.$ Аналогично $V_2\frac{dn_2}{dt}=DS\frac{n_1-n_2}{l}$

\noindent Тогда $$\frac{d(n_1-n_2)}{dt}=-\frac{n_1-n_2}{l} \frac{V_1+V_2}{V_1V_2}.$$

\noindent Проинтегрируем и получим, что 
$$n_1-n_2=(n_1-n_2)_0 e^{-t/\tau},$$ где $(n_1-n_2)_0$ 
\noindent — разность концентраций в начальный момент времени, 
$$\tau=\frac{V_1V_2}{V_1+V_2}\frac{l}{SD}.$$

\noindent Для измерения концентраций в данной установке применяются датчики теплопроводности $Д_1$, $Д_2$ (см. рис. 1) используется зависимость теплопроводности газовой смеси от ее состава. Для измерения разности концентраций газов используется мостовая схема (рис. 1). Здесь $Д_1$ и $Д_2$ — датчики теплопроводности, расположенные в сосудах $V_1$ и $V_2$. Сопротивления $R_1, R_2$ и $R$ служат для установки прибора на нуль (балансировка моста). В одну из диагоналей моста включен гальванометр, к другой подключается небольшое постоянное напряжение. Мост балансируется при заполнении сосудов (и датчиков) одной и той же смесью.\\

\noindent При заполнении сосудов смесями различного состава возникает «разбаланc» моста. При незначительном различии в составах смесей показания гальванометра, подсоединённого к диагонали моста, будут пропорциональны разности концентраций примеси. В процессе диффузии разность концентраций убывает по экспоненте, и значит по тому же закону изменяются во времени показания гальванометра 
$$U=U_0 \exp(-t/\tau).$$

\section*{Эксперементальная установка}
\noindent Схема установки изображена на рис. 1. Там же показана схема электрических соединений и конструкция многоходового крана $K_6$

\begin{figure}[h]
		    \centering
		    \includegraphics[scale=0.53]{1.png}
		    \label{Fig1}
		    \caption{Схема установки}
		\end{figure}
\noindent Установка состоит из двух сосудов $V_1$ и $V_2$ соединенных краном $К_3$, форвакуумного насоса Ф.Н. с выключателем $Т$, вакуумметра $В$ и системы напуска гелия, включающей в себя краны $К_6$ и $К_7$. Кран $К_5$ позволяет соединять установку с атмосферой. Сосуды $V_1$ и $V_2$ и порознь и вместе можно соединять с системой напуска гелия, форвакуумным насосом и с атмосферой. Для этого служат краны $К_1$, $К_2$, $К_4$ и $К_5$. Вакуумметр $В$ регистрирует давление газа, до которого заполняют тот или другой сосуды. Для сохранения гелия, а также для уменьшения неконтролированного попадания гелия в установку (по протечкам в кране $К_6$) между трубопроводом подачи гелия и краном $К_6$ поставлен металлический кран $К_7$. Его открывают только на время непосредственного заполнения установки гелием. Все остальное время он закрыт.\\

\noindent В силу того, что в сосуд требуется подавать малое давление гелия,между кранами $К_7$ и $К_4$ стоит кран $К_6$, снабженный дозатором. Дозатор - это маленький объем, который заполняют до давления гелия в трубопроводе, а затем уже эту порцию гелия с помощью крана $К_6$ впускают в установку.\\

\noindent Описание схемы электрического соединения. $Д_1$ и $Д_2$ — сопротивления проволок датчиков парциального давления, которые составляют одно плечо моста. Второе плечо моста составляют сопро- тивления $r_1$, $R_1$ и $r_2$, $R_2$. $r_1 \ll R_1$, $r_2 \ll R_2$, $R_1$ и $R_2$ спаренные, их подвижные контакты находятся на общей оси. Оба они исполь- зуются для грубой регулировки моста. Точная балансировка моста выполняется потенциометром R. Последовательно с гальванометром $Г$, стоящим в диагонали моста, поставлен магазин сопротивлений $MR$. Когда мост балансируют, магазин сопротивлений выводят на ноль. В процессе же составления рабочей смеси в сосудах $V_1$ и $V_2$ мост разбалансирован. Чтобы не сжечь при этом гальванометр, магазин $MR$ ставят на максимальное сопротивление.

\section*{Ход работы}

\noindent Включим питание электрической схемы установки рубильником $B$. Откроем краны $К_1$, $K_2$, $К_3$. Перепишем параметры установки: 
$$V_1 = V_2 = V = 775 \pm 10 \; см^{3}, \; \frac{L}{S} = 5.3 \pm 0.1 \;см^{-1}$$

\noindent Поскольку вакуумметр измеряет разность давления внутри резервуаров с атмосферным в $\frac{кгс}{cм^2}$ необходимо записать показание вакуумметра при полностью откачанном сосуде $P_0 = 99.0 \;\frac{кгс}{cм^2}$ и в дальнейшем постоянно вычитать из него показания прибора. Таким образом мы находим давление внутри установки.\\

\noindent Очистим установку от всех газов, которые в ней есть. Для этого включим форвакуумный насос (Ф.Н.) выключателем $Т$, находящемся на насосе, следом откроем $K_4$. Откачаем установку до давления $\approx 0.1 \; торр $, что достигается непрерывной работой насоса в течение 3–5 минут. Для прекращения откачки закрываем ключ $K_4$, следом выключаем насос.\\

\noindent Напустим в установку воздух до рабочего давления (вначале $P \approx 40 \; торр$), чтобы сбалансировать мост на рабочем давлении. Для этого плавно открываем $K_5$. Эту операцию повторим несколько раз, пока не будет достигнуто нужное давление. Сбалансируем мост, сначала вращая ручку $"Грубо"$, а затем $"Точно"$, устанавливая таким образом околонулевую силу тока. Заполним установку рабочей смесью согласно порядку предложенному в указании к работе: в сосуде $V_2$ должен быть воздух, а в сосуде $V_1$ — смесь воздуха, с гелием. \\

\noindent Чтобы это сделать, сначала откачиваем установку до $\sim 0.1$ торр. Закроем краны $К_2$ и $К_3$, открыв при этом краны $К_1$ и $К_4$. Заполним объём $V_1$ гелием до давления $0.1\ P_{рабочее}$, т. е. $\sim$4 торр. Давление гелия в трубопроводе немного выше атмосферного и, следовательно, много больше требуемых 4 торр, поэтому гелием наполняем сначала дозатор (маленький объём), а уже потом дозатор соединяется с полостью установки. Эту операцию проводим с помощью крана $К_6$, поворачивая его рукоятку из положения $I$ в положение $II$ и обратно до тех пор, пока не будет достигнуто нужное давление. После заполнения ёмкости $V_1$ гелием закрываем кран $К_1$, а из патрубков установки откачиваем оставшийся гелий до $\sim 0.2 \ P_{рабочее}$. Откроем кран $К_2$ и с помощью крана $К_5$ заполняем объём $V_2$ воздухом до давления $\sim 1.675 P_{pабочее}$. После этого закрываем кран К4.\\

\noindent Уравниваем давление в объёмах $V_1$ и $V_2$, открыв кран $К_1$ при уже открытом кране $К_2$. Ждем около 30 секунд, потому что в одном сосуде происходит адиабатическое сжатие, а в другом разрежение. Необходимо, чтобы исходное состояние было изотермическим. \\

\noindent Проведём измерения. Для этого откроем кран $К_3$, включим компьютер
и затем скачаем из него данные показаний гальванометра с течением времени. Процесс измерений продолжим до тех пор, пока разность концентраций (показания гальванометра) не упадет на $30–50\%.$ Будем продолжать аналогичные измерения при различных значениях $P_{рабочее}$ в интервале $40–280 \; торр$. Данные представлены в таблице 1. Давления там приведены уже в торр.\\

\noindent Для каждого из давлений построим графики, откладывая по оси абсцисс время, а по оси ординат --- логарифм от показаний гальванометра. Видим, что теоретическая зависимость $U = U_0 \cdot exp(\frac{-t}{\tau})$ подтверждается эксперементально.


%Здесь должен быть график

\begin{table}[h!]
	\begin{center}
	\begin{tabular}{|c|c|}
	\hline
	P, торр & $\ln{\frac{U}{U_0}}(t)$\\
	\hline
		40  &  $Нужны\ данные$ \\ 
		\hline
		80 & $Нужны\ данные$ \\ 
		\hline
		200 & $Нужны\ данные$ \\
		\hline
		280 & $Нужны\ данные$ \\
		\hline
 \end{tabular}
 \caption{Уравнения прямых}
\end{center}
\end{table}

По угловым коэффициентам экспериментальных прямых и известным параметрам установки рассчитаем коэффициенты взаимной диффузии и их погрешности при выбранных давлениях по формулам: $$D =  \frac{1}{2} kV \frac{L}{S}, \; \sigma_D = D \sqrt{\big(\frac{\sigma_k}{k}\big)^2 + \big(\frac{3\sigma_V }{2V}\big)^2 + \big(\frac{\sigma_{L/S}}{L/S}\big)^2},$$ где $k$ - коэффициенты наклонов прямых. Данные представлены в таблице 3.
 
 \begin{table}[!htb]
 	\centering
 	\begin{minipage}{0.45\linewidth}
 		\centering
 	\caption{}
 	\begin{tabular}{|l|l|l|}
 		\hline
 		\label{tb1}	
 		$P, торр$ & $D, \frac{см^2}{c}$ & $\sigma_D, \frac{см^2}{c}$ \\ \hline
 		40 &  8.88& 0.18 \\ 
 		\hline
 		100 & 5.50  & 0.11 \\ 
 		\hline
 		140 & 4.13 & 0.09 \\ 
 		\hline
 		180& 3.21 & 0.07 \\ 
 		\hline
 	\end{tabular}
\end{minipage}
\begin{minipage}{0.45\linewidth}
	\centering
	\caption{}
	\begin{tabular}{|c|c|c|}
		\hline
		\label{tb1}
		
		$\frac{1}{P} , торр^{-1}\cdot 10^{-3} $ & $\sigma_{\frac{1}{P}} ,\;торр^{-1}\cdot 10^{-3}$ & $D, \frac{см^2}{c} $\\ \hline
		16.67 & 2 & 8.88\\ 
		\hline
		10.00 & 0.8 &5.50 \\ 
		\hline
		7.14 & 0.4 &4.13 \\ 
		\hline
		5.56& 0.2 &3.21 \\ 
		\hline
	
	\end{tabular}
\end{minipage}
\end{table}

 Построим график зависимости коэффициента диффузии от давления в координатах $D, \; \frac{1}{P}.$ Погрешность расчитывается по формуле $\sigma{_\frac{1}{P}} = \frac{\sigma_P}{P^2}$, где $\sigma_P = 7.4 \; {торр}$. Рассчитаем величину коэффициента диффузии при атмосферном давлении. Для этого экстраполируем зависимость $D(\frac{1}{P})$ и посмотрим через какую точку проходит наша прямая. Итак, $$D_{атм} =0.68 \pm 0.06  \; \frac{cм^2}{c}.$$ Погрешность $D_{атм}$ была оценена с помошью экстраполяции крайних уравнений прямых. Табличное значение для этого коэффициента $$D_{табл} = 0.57\;\frac{cм^2}{c}.$$


\begin{center}
\begin{figure}[h!]
\begin{tikzpicture}[scale = 1.5]
\begin{axis}[
axis lines = left,
legend style={at={(1,0.21)}},
ylabel = {$D,\frac{cм^2}{c}$},
xlabel = {$ \frac{1}{P} ,  {торр}^{-1} \cdot 10^{-3}$},
minor grid style={black, line width=0.05pt},
major grid style={solid,black, line width=0.3pt},
xmin=0, xmax=20,
ymin=0, ymax=10,
ymajorgrids = true,
xmajorgrids = true,
yminorgrids = true,
xminorgrids = true,
minor tick num = 4
]
\addplot+[only marks ] plot[error bars/.cd, x dir=both, x explicit]
coordinates {
	(16.67,8.88) +-(2,2) 
	(10.0,5.5) +-(0.8,0.8)
	(7.34, 4.13) +-(0.4,0.4)
	(5.76, 3.21)+-(0.2,0.2)
};
\addplot+[red,only marks ] plot[error bars/.cd, y dir=both, y explicit]
coordinates {
	(1.3158,0.74)
};
%\addplot[blue, domain=0:20]{0.448  + 0.5062*x};
\addplot[blue, domain=0:20]{0.0  + 0.54153*x};
\legend{
	$D = (514 \pm 47) \frac{1}{P}$,$D_{атм}$
};

\end{axis}
\end{tikzpicture}
\caption{График зависимости $D(\frac{1}{P}$)}
\end{figure}
\end{center}

 Оценим по полученным результатам длину свободного пробега и размер молекулы.  Для этого воспользуемся следующими формулами. $$D = \frac{1}{3} \lambda \langle v \rangle,\; где \langle v \rangle = \sqrt{\frac{8RT}{\pi \mu}}, \; \Pi \approx \frac{kT}{\sqrt{2}\lambda P},$$ где $\Pi$ - площадь эффективного сечения частиц, $r \approx \frac{1}{2}\sqrt{\frac{\Pi}{\pi}}$.

Итак, $$\lambda \approx 1.6 \cdot 10^{-7} \; м, \; \Pi \approx 1.8\cdot 10^{-19} \; м^2, \; r \approx 1.2 \cdot 10^{-10}\; м.$$ Табличное значение для размера молекулы $r = 1.0 \cdot 10^{-10}\; м.$

\section*{Вывод}
\noindentВ данной работе было проверено, что закон $U = U_0 \cdot e^{\frac{-t}{\tau}}$ выполняется с высокой точностью.\\

\noindent Так же в работе было найдено значение коэффициента диффузии гелия в воздухе, а так же оценены длина свободного пробега и размер молекулы гелия. К сожалению, результат не сошёлся с табличным. Этому есть несколько разумных объяснений. Вероятно, были совершены ошибки на этапе заготовки смесей газов в сосудах (погрешности определения давлений) и ошибки экспериментатора. А так же есть подозрение, что газ, который в работе назывался гелием, на самом деле был разбавленным. Так же не стоит забывать о том, что численный множитель $\frac{1}{3}$ в формуле для длины свободного пробега, является на самом деле данью традиции, поэтому на ювелирную точность определения параметров в последнем пункте претендовать не стоит. В любом случае методика данной работы позволяет с достаточной точностью получить приблизительное значение коэффициента диффузии и позволяет оценить такие важные физические величины, как длина свободного пробега и размер молекулы газа.

%Вывод надо делать позже, после того, как будут данные и график
\end{document}
